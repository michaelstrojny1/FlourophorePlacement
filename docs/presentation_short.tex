\documentclass[aspectratio=169]{beamer}
\usetheme{Madrid}
\usecolortheme{default}

\usepackage{amsmath}
\usepackage{amssymb}
\usepackage{graphicx}
\usepackage{booktabs}

\title{Physics-Informed Neural Networks for Optimal Sensor Placement}
\subtitle{Nanomaterial Characterization}
\author{Michael Strojny}
\date{\today}

\begin{document}

\frame{\titlepage}

\begin{frame}{Problem \& Solution}
\textbf{Challenge:} Optimal fluorophore sensor placement for nanomaterial parameter estimation

\vspace{0.3cm}
\textbf{Our Approach:}
\begin{itemize}
    \item Physics-Informed Neural Networks (PINNs) for forward modeling
    \item Variance-based optimization for sensor placement
    \item \textbf{Result:} 70-77\% improvement over random placement
\end{itemize}

\vspace{0.3cm}
\textbf{Governing PDE:} $-D \nabla^2 u + \alpha u = f(x,y)$
\begin{itemize}
    \item $u$: concentration field, $D$: diffusion, $\alpha$: reaction
    \item Dual Gaussian sources, zero boundary conditions
\end{itemize}
\end{frame}

\begin{frame}{Methodology}
\textbf{Two-Phase Algorithm:}

\textbf{Phase 1: Multi-Reference PINN Training}
\begin{itemize}
    \item Train PINNs for sampled $(D, \alpha)$ parameter sets
    \item MLP: width=128, depth=6, tanh activation
    \item Loss: PDE residual + boundary conditions
\end{itemize}

\textbf{Phase 2: Sensor Optimization}
\begin{itemize}
    \item Maximize variance in sensor readings: $J = \frac{1}{N}\sum_k \text{Var}(\mathbf{y}_k)$
    \item Add coverage \& anti-clustering penalties
    \item Adam optimizer with sigmoid constraints
\end{itemize}
\end{frame}

\begin{frame}{Results}
\begin{table}
\centering
\begin{tabular}{@{}lcccc@{}}
\toprule
\textbf{Scenario} & \textbf{Sensors} & \textbf{Optimized J} & \textbf{Random J} & \textbf{Improvement} \\
\midrule
S1 & 16 & -0.01200 & -0.01703 & 76.1\% \\
S2 & 12 & -0.01288 & -0.02028 & 74.1\% \\
S3 & 20 & -0.01101 & -0.01551 & 77.4\% \\
S4 & 10 & -0.01348 & -0.02091 & 71.6\% \\
\bottomrule
\end{tabular}
\end{table}

\vspace{0.3cm}
\textbf{Key Findings:}
\begin{itemize}
    \item Consistent 70-77\% improvement across all scenarios
    \item Sensors avoid symmetric patterns for better coverage
    \item Method scales efficiently (2-5 minutes per scenario)
\end{itemize}
\end{frame}

\begin{frame}{Lab Implementation}
\textbf{Translation to Physical Experiments:}
\begin{enumerate}
    \item Run optimization: \texttt{python src/nano\_examples.py}
    \item Scale coordinates: $[0,1]^2 \rightarrow$ physical dimensions
    \item Place fluorescent detectors at optimized positions
    \item Measure steady-state intensities
    \item Estimate $(D, \alpha)$ via inverse PINN fitting
\end{enumerate}

\vspace{0.3cm}
\textbf{Expected Benefits:}
\begin{itemize}
    \item Improved parameter estimation accuracy
    \item Reduced experimental time and cost
    \item Better nanomaterial characterization
\end{itemize}
\end{frame}

\begin{frame}{Conclusion}
\textbf{Contributions:}
\begin{itemize}
    \item Novel variance-based design criterion for PINN sensor optimization
    \item Demonstrated 70-77\% improvement over random placement
    \item Direct applicability to nanomaterial characterization experiments
    \item Open-source implementation with reproducible results
\end{itemize}

\vspace{0.5cm}
\textbf{Future Work:} 3D optimization, multi-species systems, real-time adaptation

\vspace{0.5cm}
\centering
\textbf{Repository:} \texttt{github.com/your-username/FlourophorePlacement}

\textbf{Thank you!}
\end{frame}

\begin{frame}{References}
\footnotesize
\begin{itemize}
  \item Raissi, M., et al. \emph{Physics-Informed Neural Networks}. J. Comput. Phys. 378 (2019): 686–707.
  \item Atkinson, A.C., et al. \emph{Optimum Experimental Designs}. Oxford University Press (2007).
  \item Baydin, A.G., et al. \emph{Automatic Differentiation in Machine Learning}. J. Mach. Learn. Res. 18(1) (2018).
\end{itemize}
\end{frame}

\end{document}
